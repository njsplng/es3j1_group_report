\documentclass[11pt, onecolumn]{article}
\usepackage{geometry}
\usepackage{graphicx}
\usepackage{amssymb}
\usepackage{microtype}
\usepackage{wrapfig}
\usepackage{amsmath}
\usepackage{indentfirst}
\usepackage{setspace}
\usepackage[format=plain, font=it]{caption}
\usepackage{todonotes}
\usepackage[hidelinks]{hyperref}
\usepackage{lipsum}

\geometry{a4paper, margin=2.5cm}
\singlespacing
\title{ES3J1 Group 3 Report}
\author{Anya Akram, Edward Stanley, Favour Rabiu, Matt Brooks, Nojus Plungė}
\date{\today}

\begin{document}
\pagenumbering{arabic}
\maketitle

\section*{Question 1}
\par \textit{For your report, you are required to describe all the steps you used to obtain the final model of the DC motor in details. Include any results you deem essential to support your findings.}
\noindent\makebox[\linewidth]{\rule{\textwidth}{0.4pt}}
\par Paragraph
\par As the output produced by the motor is not smooth, to obtain its model that can be analysed using control system theory, a filter needs to be designed.
\par The filter (and thus, the model) tuning was done in the following order:
\begin{enumerate}
    \item The filter coefficient was set to the simplest low-pass transfer function $\frac{1}{s+1}$ to start smoothing the motor output, obtained by applying the step input. The filter value tuning process, outlined in the steps below was started to obtain an accurate filter, representing .
    \item The motor circuit, as set up in the brief, was assembled; the motor and Arduino were connected to the circuit and a link was established to the Simulink workspace.
    \item The motor was run using the MATLAB hardware add-on. The signal could be traced as going from the workspace input, through the PWM converter block and into the Arduino. The PWM signal generated by the Arduino then flowed through the circuit to power the motor, which had its encoder linked to the Arduino – this signal was then converted into motor RPM in the workspace and the obtained results were designated as the output. The input into the system and the output of the motor (in RPM) was saved to the workspace for further analysis.
    \item System Identification Toolbox (SIT) was used, with the input being the workspace signal and the output being the motor's revolutions per minute (RPM).
    \item In SIT the analysis was set for the transfer function to have no zeroes and one pole – this was chosen as the transfer function graph output from the unsmoothed output, by inspection, approximates to a first order transfer function output with step input.
    \item The obtained transfer function was scaled by dividing both the numerator and the denominator by step time.
    \item The obtained transfer function was plotted against the unfiltered motor output to verify the model accuracy and legitimacy. If proven incorrect, the design process was repeated until a fitting filter that produced a smooth transfer function was obtained.
\end{enumerate}

\par The final design allowed for the produced output to be of a smooth 1st order transfer function response, which enabled further analysis and tuning of the system.
\section*{Question 2}
\par \textit{Describe all the steps that you used in obtaining the PI controller gains, (i.e. $K_p$ and $K_i$). It is highly encouraged to implement the PI controller manually instead of using the built-in PID Controller block.}
\par \textit{Include any results you deem essential to support your findings. Remember to take into practical consideration of the final controller gains that you will implement on the Arduino Uno. In other words, while in simulation, you can set the gains to a large value to get good performance, whether this large gain values are feasible or not to be implemented on the Arduino Uno is another matter.}
\noindent\makebox[\linewidth]{\rule{\textwidth}{0.4pt}}
\par \textbf{TEXT GOES HERE}
\section*{Question 3}
\par \textit{To fully test the performance of your PI controller, the Desired Motor Speed (Reference) should cover a substantial range of practical motor speed. Critically comment on your results. Include any results you deem essential to support your findings. Also, do remember to comment on the expectation (Task 2) versus reality (Task 3) of the implementation.}
\noindent\makebox[\linewidth]{\rule{\textwidth}{0.4pt}}
\par \textbf{TEXT GOES HERE}
\section*{Question 4}
\par \textit{Using what you have learned throughout this module (and beyond such as non- linearities effect, etc), try to improve the performance of the DC motor whether from the modelling side or the control design side. Feel free to explore and be adventurous in this exercise (as long as the safety precaution is adhered to ensure you do not damage the DC motor and the Arduino Uno). If required, revisit back all the steps from Tasks 1 to 3. Remember that in an actual engineering work, there is always room for improvement and the process is always iterative.}
\noindent\makebox[\linewidth]{\rule{\textwidth}{0.4pt}}
\par \textbf{TEXT GOES HERE}
\end{document}